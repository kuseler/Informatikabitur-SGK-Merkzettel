\section{XML}
\begin{zitat}{XML}
XML steht für  "'eXtensible Markup Language"'. XML ist eine universelle, erweiterbare Sprache, mit der man konkrete Auszeichnungssprachen erzeugen kann.
\end{zitat}
.\\
Mit XML können Dokumente zur Informationsdarstellung gebildet werden.
Die Struktur von XML-Dokumenten kann genau festgelegt werden, um Sprachen für bestimmte Anwendungsbereiche zu entwickeln.
\section{Wohlgeformt}
\begin{zitat}{Wohlgeformt}
 Ein XML-Dokument, das alle Regeln von XML erfüllt, heißt wohlgeformt.
\end{zitat}
.\\
Beispiel-XML-Dokument:
\lstset{language=XML}
\lstset{backgroundcolor=\color{block-gray}}
\begin{lstlisting}
<?xml version="1.0" encoding="utf-8"?>
<!DOCTYPE Kurs SYSTEM "kurs.dtd">
<Kurs name="Infokurs">
    <Schueler id="87" geschlecht="m">
      <Vorname>Tobias</Vorname>
      <Name>Schwarz</Name>
      <Kurssprecher/>
    </Schueler>
</Kurs>
\end{lstlisting}
.\\
Ein XML-Dokument beginnt mit einem Prolog, dieser kennzeichnet das Dokument als XML-Dokument.
\begin{lstlisting}
<?xml version="1.0" encoding="utf-8"?>
\end{lstlisting}
.\\
Eine XML-Datei besteht aus Elementen. Diese können sowohl untergeordnete Elemente (in diesem Fall Name und Vorname) als auch Attribute (id und geschlecht) halten. Sogar leere Elemente sind möglich (Kurssprecher).
\begin{lstlisting}
<Schüler id="87" geschlecht="m">
  <Name>Schwarz</Name>
  <Vorname>Tobias</Vorname>
  <Kurssprecher/>
</Schüler>
\end{lstlisting}
.\\
Eine Dokumenttypdefinition kann wie folgt eingebunden werden:
\begin{lstlisting}
<!DOCTYPE Kurs SYSTEM "kurs.dtd">
\end{lstlisting}
\newpage
\section{Dokumenttypdefinitionen (DTD)}
\begin{zitat}{Valide}
Ein XML-Dokument, das alle Festlegungen einer DTD erfüllt, heißt gültig bzw. valide bzgl. dieser DTD.
\end{zitat}
Die DTD für das oben angegebene Beispiel wäre folgendes:
\begin{lstlisting}
<!ELEMENT Kurs (Schueler*) >
<!ATTLIST Kurs
  name CDATA #REQUIRED
>
<!ELEMENT Schueler (Vorname, Name, Kurssprecher?) >
<!ATTLIST Schueler
  id CDATA #REQUIRED
  geschlecht CDATA #REQUIRED
>
<!ELEMENT Vorname (#PCDATA) >
<!ELEMENT Name (#PCDATA) >
<!ELEMENT Kurssprecher EMPTY >
\end{lstlisting}
.\\
In Attributlisten muss bei CDATA spezifiziert werden, ob das Attribut in in jedem Element vorkommen muss (REQUIRED) oder weggelassen werden kann (IMPLIED).
\begin{table}[h]
\caption{Zusatzsymbole}
\begin{tabular}{ll}
() & Klammern zur Bildung von Elementgruppen                                   \\
,  & Trennzeichen innerhalb einer Sequenz von Elementen                        \\
|  & Trennzeichen zwischen sich ausschließenden Alternativen                   \\
*  & Element(gruppe) kann beliebig oft (auch gar nicht) vorkommen              \\
+  & Element(gruppe) muss mindestens einmal vorkommen, kann mehrfach vorkommen \\
?  & Element(gruppe) kann einmal oder kein mal vorkommen
\end{tabular}
\end{table}
